\documentclass[fontset=ubuntu]{ctexart}


% ------- 导入包 ------- %
\usepackage{hyperref}
\usepackage{fancyhdr}
\usepackage{tabularx}
\usepackage{graphicx}
\usepackage{amsmath}
\usepackage{multicol}

% ------- 页眉页脚设置 ------- %
\pagestyle{fancy}
\fancyhf{}
\fancyfoot[C]{\thepage}
\renewcommand{\headrulewidth}{0pt}
\renewcommand{\footrulewidth}{0pt}


% ------- 图片编号 ------- %

% ------- 标题页 ------- % 
\title{红酒质量分析报告}
\author{回归分析小组}
\date{\today}


% ------- 主文档 ------- %
\begin{document}

    % ------- 标题页 ------- %
    \maketitle
    \thispagestyle{empty}
    \newpage

    % ------- 目录页 ------- %
    \tableofcontents
    \thispagestyle{empty}
    \newpage
    \setcounter{page}{1} % 重置页码

    % ------- 正文 ------- %
    \section{问题背景}
        随着酒类产品在市场上的广泛受欢迎,对于酒的质量和特征的深入了解变得至关重要。为了更好地了解酒的品质,我们进行了一项回归分析,重点关注了酒的酸度,二氧化硫($SO_2$)含量等特征。这些特征在很大程度上影响了酒的口感、风味和保存能力。 

    \section{数据说明}
        我们通过 \href{https://archive.ics.uci.edu/dataset/186/wine+quality}{UC Irvine}仓库下载了红酒数据,数据集中的特征包含酒的各类化学成分指标以及酒的品质评价。其中,成分指标,如$pH$值、$SO_2$含量、残糖量等通过物理化学检测得出;酒的品质由专业品酒师做出评价。(每个样本由三个品酒师做出评价,每个人的评分为$0$(差)到$10$(好)的一个整数,最终评价取三人的中位数)。红酒共$599$条数据,其中不含缺失值,共记录了$12$个特征,特征说明如 \ref{tab:features} 所示。

        \begin{table}[htbp]
            \centering
            \caption{酒的特征说明}
            \vspace{5pt}
            \begin{tabular}{cccc}
                \hline
                变量名 & 中文含义 & 变量类型 & 单位 \\
                \hline
                fixed acidity & 固定酸度 & 连续型变量 & g/L \\
                Volatile Acidity & 挥发性酸度 & 连续型变量 & g/L \\
                Citric Acid & 柠檬酸 & 连续型变量 & g/L\\
                Residual Sugar & 残糖 & 连续型变量 & g/L \\
                Chlorides & 氯化物 & 连续型变量 & g/L\\
                Free Sulfur Dioxide & 游离二氧化硫 & 连续型变量 & g/L \\
                Total Sulfur Dioxide & 总二氧化硫 & 连续型变量 & g/L \\
                Density & 密度 & 连续型变量 & g/mL \\
                pH &  葡萄酒的pH值 & 连续型变量 & \\
                Sulphates & 硫酸盐 & 连续型变量 & g/L \\
                Alcohol & 醇度 & 连续型变量 & \% \\
                Quality & 酒品 & 离散型变量 & \\ 
                \hline
            \end{tabular}
            \label{tab:features}
        \end{table}

    \section{描述性统计}
        \subsection{数值特征}
        对数据的初步描述如表 \ref{tab:description1-4},表\ref{tab:description5-8}和表\ref{tab:description9-12} 所示,包含平均值、最小值、最大值、中位数。
        \begin{table}[ht]
            \centering
            \caption{数据描述1-4}
            \vspace{5pt}
            \begin{tabular}{llll}
                \hline
                fixed.acidity & volatile.acidity &  citric.acid & residual.sugar \\ 
                \hline
                Min.   : 6.557   & Min.   :0.0052   & Min.   :-0.1331   & Min.   :1.223   \\ 
                1st Qu.: 7.857   & 1st Qu.:0.3278   & 1st Qu.: 0.3126   & 1st Qu.:1.784   \\ 
                Median : 8.622   & Median :0.4581   & Median : 0.4563   & Median :1.977   \\ 
                Mean   : 8.859   & Mean   :0.4808   & Mean   : 0.4535   & Mean   :2.407   \\ 
                3rd Qu.: 9.309   & 3rd Qu.:0.6300   & 3rd Qu.: 0.5803   & 3rd Qu.:2.457   \\ 
                Max.   :12.642   & Max.   :1.0396   & Max.   : 1.1742   & Max.   :6.780   \\ 
                \hline
            \end{tabular}
            \label{tab:description1-4}
        \end{table}

        \begin{table}[ht]
            \centering
            \caption{数据描述5-8}
            \vspace{5pt}
            \begin{tabular}{llll}
                \hline
                chlorides & free.sulfur.dioxide & total.sulfur.dioxide &    density \\ 
                \hline
                Min.   :-0.17986   & Min.   : 2.872   & Min.   :  9.874   & Min.   :0.7226   \\ 
                1st Qu.: 0.07193   & 1st Qu.: 8.041   & 1st Qu.: 24.112   & 1st Qu.:0.9328   \\ 
                Median : 0.16598   & Median :15.930   & Median : 56.041   & Median :0.9960   \\ 
                Mean   : 0.21012   & Mean   :16.475   & Mean   : 63.791   & Mean   :1.0014   \\ 
                3rd Qu.: 0.35014   & 3rd Qu.:22.089   & 3rd Qu.:102.992   & 3rd Qu.:1.0719   \\ 
                Max.   : 0.76331   & Max.   :39.160   & Max.   :151.197   & Max.   :1.2901   \\ 
                \hline
            \end{tabular}
            \label{tab:description5-8}
        \end{table}

        \begin{table}[ht]
            \centering
            \caption{数据描述9-12}
            \vspace{5pt}
            \begin{tabular}{llll}
                \hline
                pH &   sulphates &    alcohol &    quality \\ 
                \hline
                Min.   :2.611   & Min.   :-0.09304   & Min.   : 8.768   & Min.   :4.000   \\ 
                1st Qu.:3.017   & 1st Qu.: 0.34213   & 1st Qu.: 9.273   & 1st Qu.:5.000   \\ 
                Median :3.134   & Median : 0.50100   & Median : 9.498   & Median :5.000   \\ 
                Mean   :3.129   & Mean   : 0.64124   & Mean   : 9.888   & Mean   :5.581   \\ 
                3rd Qu.:3.256   & 3rd Qu.: 0.72854   & 3rd Qu.:10.346   & 3rd Qu.:6.000   \\ 
                Max.   :3.582   & Max.   : 2.84062   & Max.   :12.465   & Max.   :8.000   \\ 
                \hline
            \end{tabular}
            \label{tab:description9-12}
        \end{table}
        
        \clearpage
        \subsection{因变量描述}
            首先通过直方图观察酒品分布情况,可以发现,数据主要集中在$5, 6, 7$。 
            \begin{figure}[htbp]
                \centering
                \includegraphics[width=0.8\textwidth]{../figures/quality-frequency.png}
                \label{fig:quality}
                \caption{酒品数据分布}
            \end{figure}

        \subsection{自变量描述}
            我们通过自变量与酒品的箱线图对其进行描述。下面展示部分具有代表性的数据。
            \clearpage
            \subsubsection{固定酸度(Fixed Acidity)}
                固定酸度关于酒品的箱线图如图所示。
                \begin{figure}[htbp]
                    \centering
                    \includegraphics[width=0.8\textwidth]{../figures/fixed.acidity-plot.png}
                    \label{fig:fixed-acidity}
                    \caption{固定酸度箱线图}
                \end{figure}

            \subsubsection{挥发性酸度(Volatile Acidity)}
                挥发性酸度关于酒品的箱线图如图所示。
                \begin{figure}[htbp]
                    \centering
                    \includegraphics[width=0.8\textwidth]{../figures/volatile.acidity-plot.png}
                    \label{fig:volatile-acidity}
                    \caption{挥发性酸度分布箱线图}
                \end{figure}
            
            \subsubsection{游离二氧化硫(Free Sulfur Dioxide)}
                游离二氧化硫含量关于酒品的箱线图如图所示。
                \begin{figure}[htbp]
                    \centering
                    \includegraphics[width=0.8\textwidth]{../figures/free.sulfur.dioxide-plot.png}
                    \label{fig:free.sulfur.dioxide}
                    \caption{游离二氧化硫箱线图}
                \end{figure}   

            \subsubsection{醇度(Alcohol)}
                酒精含量关于酒品的箱线图如图所示。
                \begin{figure}[htbp]
                    \centering
                    \includegraphics[width=0.8\textwidth]{../figures/alcohol-plot.png}
                    \label{fig:alcohol}
                    \caption{醇度箱线图}
                \end{figure} 
        
        \subsection{自变量相关性分析}
            我们对自变量相关性进行分析,结果如图所示。可以发现,与体现酒的酸碱性的自变量相关性相对较大,需要通过后续处理消除它们的复共线性。
            \begin{figure}[htbp]
                \centering
                \includegraphics[width=0.8\textwidth]{../figures/covariance.png}
                \label{fig:covariance}
                \caption{自变量相关性热力图}
            \end{figure}
    \section{数据建模}
        我们首先使用全模型进行建模,通过红酒的化学成分特征,对酒品进行回归分析。然后使用选模型对变量进行选择,最后使用Box-cox对quality进行变换。
        \clearpage
        \subsection{全模型}
            首先,我们对数据进行全模型最小二乘回归,得到结果如下:
            \begin{table}[htbp]
                \centering
                \begin{tabular}{rrrrl}
                    \hline
                    & Estimate & Std. Error & t value & Pr($>$$|$t$|$) \\ 
                    \hline
                    (Intercept) & 1.5577 & 0.8651 & 1.80 & 0.0723 . \\ 
                    fixed.acidity & 0.0771 & 0.0220 & 3.51 & 0.0005 *** \\ 
                    volatile.acidity & -1.0448 & 0.1477 & -7.07 & 0.0000 *** \\ 
                    citric.acid & 0.0886 & 0.1538 & 0.58 & 0.5648 \\ 
                    residual.sugar & 0.0100 & 0.0277 & 0.36 & 0.7174 \\ 
                    chlorides & -0.2167 & 0.1891 & -1.15 & 0.2523 \\ 
                    free.sulfur.dioxide & -0.0181 & 0.0044 & -4.10 & 0.0000 *** \\ 
                    total.sulfur.dioxide & -0.0008 & 0.0012 & -0.65 & 0.5157 \\ 
                    density & 0.1462 & 0.2437 & 0.60 & 0.5488 \\ 
                    pH & -0.2349 & 0.2072 & -1.13 & 0.2574 \\ 
                    sulphates & -0.0626 & 0.0550 & -1.14 & 0.2556 \\ 
                    alcohol & 0.4856 & 0.0376 & 12.91 & 0.0000 *** \\ 
                    \hline
                    \multicolumn{5}{c}{Residual standard error: 0.5762 on 547 degrees of freedom} \\
                    \multicolumn{5}{c}{Multiple R-squared:  0.5656,	Adjusted R-squared:  0.5569} \\
                    \multicolumn{5}{c}{F-statistic: 64.76 on 11 and 547 DF,  p-value: < 2.2e-16} \\
                    \hline
                \end{tabular}
                \label{tab:linear-model}
            \end{table}
            
            结果显示,在0.05的置信水平下,固定酸度、挥发性酸度、游离二氧化氮和醇度与酒品呈正相关。
            \clearpage
            然后我们对模型进行回归诊断,结果如下图:

            \begin{figure}[htbp]
                \centering
                \includegraphics[width=0.8\textwidth]{../model-summary/linear-model-plot.png}
                \label{fig:linear-model}
                \caption{全模型诊断图}
            \end{figure}

            在左上角残差图中,有一部分离群点;
            在右上角Q-Q图中,有一部分数据发生了偏离;
            因此有必要对数据进一步处理。

            在进一步处理之前,我们先对模型进行一系列检验。
            \clearpage
            \subsubsection{异常值检验}
                利用Cook距离检验数据集中的异常值点,结果如下:
                \begin{figure}[htbp]
                    \centering
                    \includegraphics[width=0.8\textwidth]{../figures/cook.png}
                    \label{fig:cook}
                    \caption{Cook距离}
                \end{figure}
            
                可以看到,所有数据点的Cook距离都接近0。

            \subsubsection{异方差检验}
                利用ncvTest检验模型异方差性,结果如下:
                \begin{table}[htbp]
                    \centering
                    \caption{异方差检验}
                    \vspace{5pt}
                    \begin{tabular}{ccc}
                        \hline
                        Chisquare & Df & p \\
                        \hline
                        0.1033325 & 1 & 0.74787 \\
                        \hline
                    \end{tabular}
                \end{table}
                
                检验得到p值为0.74787,因此接受原假设,认为模型不具有异方差性。
               
            \subsubsection{自相关性检验}
                利用dwtest对误差自相关性进行检验,结果如下:
                \begin{table}[htbp]
                    \centering
                    \caption{自相关性检验}
                    \vspace{5pt}
                    \begin{tabular}{cc}
                        \hline
                        DW & p \\
                        \hline
                        2.0775 & 0.8176 \\
                        \hline
                    \end{tabular}
                \end{table}
                
                检验得到p值为0.8176,因此接受原假设,认为模型不具有自相关性。

            \subsubsection{多重共线性检验}
                利用VIF对自变量共线性进行检验,结果如下:
                \begin{table}[ht]
                    \centering
                    \caption{多重共线性}
                    \vspace{5pt}
                    \begin{tabular}{rr}
                        \hline
                        & VIF \\ 
                        \hline
                        fixed.acidity & 1.50 \\ 
                        volatile.acidity & 1.45 \\ 
                        citric.acid & 1.62 \\ 
                        residual.sugar & 1.75 \\ 
                        chlorides & 2.08 \\ 
                        free.sulfur.dioxide & 2.94 \\ 
                        total.sulfur.dioxide & 4.12 \\ 
                        density & 1.04 \\ 
                        pH & 2.08 \\ 
                        sulphates & 1.49 \\ 
                        alcohol & 1.62 \\ 
                        \hline
                    \end{tabular}
                \end{table}

                若VIF小于1,表示自变量不存在多重共线性的问题;若VIF在1到5之间,表示存在轻微的多重共线性问题;若VIF大于5,表示存在较强的多重共线性。可以看到,总二氧化硫可能与其他变量的相关性相对较强。
        \subsection{选模型}
            为了解决多重共线性的问题,我们通过AIC准则和BIC准则选取变量。
            \subsubsection{AIC准则}
                使用AIC准则选择变量,结果如下:

                \begin{table}[ht]
                    \centering
                    \begin{tabular}{rrrrl}
                        \hline
                        & Estimate & Std. Error & t value & Pr($>$$|$t$|$) \\ 
                        \hline
                        (Intercept) & 0.6495 & 0.3696 & 1.76 & 0.0794 . \\ 
                        fixed.acidity & 0.0952 & 0.0190 & 5.01 & 7.46e-07 *** \\ 
                        volatile.acidity & -1.0721 & 0.1330 & -8.06 & 4.78e-15 *** \\ 
                        free.sulfur.dioxide & -0.0201 & 0.0026 & -7.66 & 8.48e-14 *** \\ 
                        alcohol & 0.4991 & 0.0329 & 15.16 & < 2e-16 *** \\ 
                        \hline
                        \multicolumn{5}{c}{Residual standard error: 0.5743 on 554 degrees of freedom} \\
                        \multicolumn{5}{c}{Multiple R-squared:  0.563,	Adjusted R-squared:  0.5598 } \\
                        \multicolumn{5}{c}{F-statistic: 178.4 on 4 and 554 DF,  p-value: < 2.2e-16} \\
                        \hline
                    \end{tabular}
                \end{table}

            \subsubsection{BIC准则}
                使用BIC准则选择变量,结果如下:

                \begin{table}[ht]
                    \centering
                    \begin{tabular}{rrrrl}
                        \hline
                        & Estimate & Std. Error & t value & Pr($>$$|$t$|$) \\ 
                        \hline
                        (Intercept) & 0.6495 & 0.3696 & 1.76 & 0.0794 . \\ 
                        fixed.acidity & 0.0952 & 0.0190 & 5.01 & 7.46e-07 *** \\ 
                        volatile.acidity & -1.0721 & 0.1330 & -8.06 & 4.78e-15 *** \\ 
                        free.sulfur.dioxide & -0.0201 & 0.0026 & -7.66 & 8.48e-14 *** \\ 
                        alcohol & 0.4991 & 0.0329 & 15.16 & < 2e-16 *** \\ 
                        \hline
                        \multicolumn{5}{c}{Residual standard error: 0.5743 on 554 degrees of freedom} \\
                        \multicolumn{5}{c}{Multiple R-squared:  0.563,	Adjusted R-squared:  0.5598 } \\
                        \multicolumn{5}{c}{F-statistic: 178.4 on 4 and 554 DF,  p-value: < 2.2e-16} \\
                        \hline
                    \end{tabular}
                \end{table}

                使用AIC准则和BIC准则筛选出来的变量相同。

        \subsection{Box-cox变换}
            
        \subsection{主成份分析}
    \section{结论及建议}

    \newpage
    %\bibliographystyle{plain}
    %\bibliography{ref} 

\end{document}