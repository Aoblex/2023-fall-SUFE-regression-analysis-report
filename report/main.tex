\documentclass{ctexart}


% ------- 导入包 ------- %
\usepackage{hyperref}
\usepackage{fancyhdr}

% ------- 页眉页脚设置 ------- %
\pagestyle{fancy}
\fancyhf{}
\fancyfoot[C]{\thepage}
\renewcommand{\headrulewidth}{0pt}
\renewcommand{\footrulewidth}{0pt}


% ------- 字体设置 ------- %

% ------- 标题页 ------- % 
\title{SEER数据分析报告}
\author{回归分析小组}
\date{\today}


% ------- 主文档 ------- %
\begin{document}

    \maketitle
    \thispagestyle{empty}
    \newpage

    \tableofcontents
    \thispagestyle{empty}
    \newpage

    \setcounter{page}{1}
    \section{问题背景}
        \subsection{肺癌}
            肺癌通常被认为是一种严重的癌症类型,因为它通常在早期没有明显的症状,
            导致在诊断时已经进展到晚期。
            由于这个原因,肺癌的治疗可能面临更大的挑战,治疗成功率相对较低。
        \subsection{对肺癌进行数据分析的重要意义}
            通过对肺癌患者的统计数据进行分析,可以识别患病的风险因素、疾病发展的模式以及不同人群之间的差异。
            这有助于预测患者的疾病风险,为预防和早期筛查提供信息。
            统计数据分析可以揭示不同治疗方法的效果,有助于优化治疗方案。
            了解患者对特定治疗的反应,可以个体化治疗计划,提高治疗的效果和患者的生存率。

    \section{数据来源}
        我们通过 \texttt{SEER*Stat} 软件下载了 $1975-2020$ 年期间的肺癌数据 \cite{seer2023}。
        \subsection{基本信息}
        \begin{table}[htbp]
            \centering
            \begin{tabular}{c}
                \hline
                变量名 \\
                \hline
                Patient ID \\
                Age recode with single ages \\
                Sex \\
                Race recode(W,B,AI,API) \\
                \hline
            \end{tabular}
            \caption{基本信息变量}
            \label{tab:basic-information}
        \end{table}
        
        \subsection{个人生活信息}
        \begin{table}[hbtp]
            \centering
            \begin{tabular}{c}
                \hline
                变量名 \\
                \hline
                Rural Urban Continuum Code \\
                Marital status at diagnosis \\
                Median household income inflation adj to 2021 \\
                \hline
            \end{tabular}
            \caption{个人生活变量}
            \label{tab:personal-information}
        \end{table}

        \subsection{}
    \section{描述性统计}

    \section{数据建模}

    \section{结论及建议}

    \newpage
    \bibliographystyle{plain}
    \bibliography{ref} 

\end{document}