\documentclass[fontset=ubuntu]{ctexart}


% ------- 导入包 ------- %
\usepackage{hyperref}
\usepackage{fancyhdr}
\usepackage{tabularx}

% ------- 页眉页脚设置 ------- %
\pagestyle{fancy}
\fancyhf{}
\fancyfoot[C]{\thepage}
\renewcommand{\headrulewidth}{0pt}
\renewcommand{\footrulewidth}{0pt}


% ------- 字体设置 ------- %

% ------- 标题页 ------- % 
\title{酒的质量分析报告}
\author{回归分析小组}
\date{\today}


% ------- 主文档 ------- %
\begin{document}

    % ------- 标题页 ------- %
    \maketitle
    \thispagestyle{empty}
    \newpage

    % ------- 目录页 ------- %
    \tableofcontents
    \thispagestyle{empty}
    \newpage
    \setcounter{page}{1} % 重置页码

    % ------- 正文 ------- %
    \section{问题背景}
        随着酒类产品在市场上的广泛受欢迎,对于酒的质量和特征的深入了解变得至关重要。为了更好地了解酒的品质,我们进行了一项回归分析,重点关注了酒的酸度,二氧化硫($SO_2$)含量等特征。这些特征在很大程度上影响了酒的口感、风味和保存能力。 

    \section{数据说明}
        我们通过 \href{https://archive.ics.uci.edu/dataset/186/wine+quality}{UC Irvine}仓库下载了红酒以及白酒数据,数据集中的特征包含酒的各类化学成分指标以及酒的品质评价。其中,成分指标,如$pH$值、$SO_2$含量、残糖量等通过物理化学检测得出;酒的品质由专业品酒师做出评价。(每个样本由三个品酒师做出评价,每个人的评分为0(差)到10(好)的一个整数,最终评价取三人的中位数)。特征说明如 \ref{tab:features} 所示。

        \begin{table}[htbp]
            \centering
            \caption{酒的特征说明}
            \vspace{5pt}
            \begin{tabular}{cccc}
                \hline
                变量名 & 中文含义 & 变量类型 & 单位 \\
                \hline
                fixed acidity & 固定酸度 & 连续型变量 & g/L \\
                Volatile Acidity & 挥发性酸度 & 连续型变量 & g/L \\
                Citric Acid & 柠檬酸 & 连续型变量 & g/L\\
                Residual Sugar & 残糖 & 连续型变量 & g/L \\
                Chlorides & 氯化物 & 连续型变量 & g/L\\
                Free Sulfur Dioxide & 游离二氧化硫 & 连续型变量 & g/L \\
                Total Sulfur Dioxide & 总二氧化硫 & 连续型变量 & g/L \\
                Density & 密度 & 连续型变量 & g/mL \\
                pH &  葡萄酒的pH值 & 连续型变量 & \\
                Sulphates & 硫酸盐 & 连续型变量 & g/L \\
                Alcohol & 醇度 & 连续型变量 & \% \\
                Quality & 酒品 & 离散型变量 & \\ 
                \hline
            \end{tabular}
            \label{tab:features}
        \end{table}

    \section{描述性统计}

    \section{数据建模}

    \section{结论及建议}

    \newpage
    %\bibliographystyle{plain}
    %\bibliography{ref} 

\end{document}